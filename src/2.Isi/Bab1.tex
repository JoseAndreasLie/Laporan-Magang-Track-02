%-----------------------------------------------------------------------------%
\chapter{\babSatu}
%-----------------------------------------------------------------------------%

%-----------------------------------------------------------------------------%
\section{Latar Belakang Masalah}
%-----------------------------------------------------------------------------%

Transformasi digital dalam pengelolaan sumber daya manusia (SDM) telah menjadi salah satu faktor utama dalam peningkatan efisiensi operasional perusahaan di era industri 4.0. Penerapan sistem informasi kepegawaian atau \textit{Human Resource Information System} (HRIS) memungkinkan proses administrasi dan pengambilan keputusan berbasis data dilakukan secara lebih cepat dan akurat \cite{moussa2020impact}. Digitalisasi fungsi SDM tidak hanya berdampak pada efisiensi kerja, tetapi juga mendorong peningkatan transparansi dan konsistensi data pegawai \cite{shiferaw2025digital}. 

Salah satu aspek penting dari digitalisasi SDM adalah otomatisasi proses penggajian atau \textit{payroll automation}. Menurut penelitian oleh Onuotu dan Amaewhule (2024), penerapan sistem akuntansi terotomatisasi seperti Square Payroll dan QuickBooks Payroll secara signifikan meningkatkan efisiensi dan akurasi dalam persiapan penggajian di kalangan usaha kecil dan menengah, dengan pengurangan kesalahan hingga 50\% dan penurunan beban kerja manual hingga 90\% \cite{onuotu2024influence}. Hal ini menunjukkan bahwa integrasi sistem penggajian digital berperan penting dalam mempercepat proses administrasi serta memastikan keakuratan data kepegawaian di berbagai sektor industri.


% Sistem ini dikembangkan menggunakan arsitektur PERN Stack (PostgreSQL, Express.js, React.js, Node.js). Pemilihan PostgreSQL didasarkan pada kemampuannya menangani relasi kompleks dan transaksi skala enterprise, sesuai dengan standar yang digunakan dalam pengembangan sistem informasi perusahaan modern \cite{arnold2019hrdbms}.


    % KONTEN BAB 3
    % Sebagai Backend Engineer yang melanjutkan pengembangan sistem CHRIS, penulis telah 
    % berkontribusi pada beberapa pengembangan krusial:

    % \begin{enumerate}
    %     \item \textbf{Pembuatan struktur \textit{Hierarchy Tree}}, yang berfungsi 
    %     untuk menetapkan struktur supervisi antarpegawai dan mempermudah logika 
    %     bisnis seperti persetujuan cuti.
        
    %     \item \textbf{Pengembangan modul Payroll dan Slip Gaji}, yang secara otomatis 
    %     menghitung gaji berdasarkan data status kepegawaian dan kehadiran, 
    %     meningkatkan efisiensi dan mengurangi kesalahan.
        
    %     \item \textbf{Refactor dan validasi ulang sistem User Management}, termasuk 
    %     pengelolaan data pegawai dan status pekerjaan untuk mendukung integrasi data 
    %     secara menyeluruh.
    % \end{enumerate}

    % Ketiga kontribusi tersebut merupakan bagian penting dalam mewujudkan sistem 
    % pengelolaan sumber daya manusia yang adaptif terhadap pertumbuhan perusahaan 
    % dan selaras dengan praktik terbaik pengembangan sistem informasi kepegawaian.

% %-----------------------------------------------------------------------------%
% \section{Permasalahan}
% %-----------------------------------------------------------------------------%
% Pada bagian ini akan dijelaskan mengenai definisi permasalahan 
% yang \saya~hadapi dan ingin diselesaikan serta asumsi dan batasan 
% yang digunakan dalam menyelesaikannya.


%-----------------------------------------------------------------------------%
\section{Maksud dan Tujuan Kerja Magang}
%-----------------------------------------------------------------------------%

Pelaksanaan kerja magang di PT Ganda Visi Jayatama bertujuan untuk memberikan pengalaman langsung kepada mahasiswa dalam menghadapi permasalahan riil di industri, serta berkontribusi terhadap pengembangan sistem informasi yang sedang berjalan. Melalui kegiatan magang ini, peserta memperoleh pemahaman yang lebih mendalam mengenai praktik terbaik dalam pengembangan perangkat lunak, khususnya dalam konteks pengelolaan sistem kepegawaian berbasis web.

Secara khusus, maksud dari kerja magang ini adalah:

\begin{itemize}
    \item Mengimplementasikan pengetahuan akademik yang telah diperoleh selama masa perkuliahan dalam lingkungan kerja profesional.
    \item Mengasah keterampilan teknis dan kolaboratif melalui kerja tim lintas divisi dalam proyek pengembangan perangkat lunak yang kompleks.
    \item Mengamati, mempelajari, dan memahami alur kerja profesional dalam pengembangan sistem backend yang terstruktur dan terdokumentasi.
\end{itemize}

Adapun tujuan utama dari pelaksanaan magang ini, yang difokuskan pada pengembangan sistem CHRIS, meliputi:

\begin{enumerate}
    \item Mengembangkan dan menyempurnakan modul \textbf{Payroll}, agar proses penggajian dapat dilakukan secara otomatis, terstandarisasi, dan efisien.
    \item Mengoptimalkan modul \textbf{Leave Permit} agar mendukung alur persetujuan berdasarkan struktur organisasi dan menyediakan fitur pembatalan cuti yang fleksibel.
    \item Meningkatkan \textbf{keamanan akses sistem} melalui penerapan autentikasi biometrik pada aplikasi mobile CHRISM.
    \item Menyusun dan menerapkan struktur \textbf{hierarki supervisi} berbasis pohon (\textit{tree hierarchy}) untuk mendukung proses persetujuan yang lebih fleksibel.
    \item Melakukan \textbf{refaktor dan validasi form input} pada modul \textbf{User Management} untuk meningkatkan akurasi dan integritas data pegawai.
    \item Mengembangkan \textbf{RESTful API} yang mendukung integrasi antara frontend dan backend secara optimal.
\end{enumerate}

Dengan pencapaian tujuan-tujuan tersebut, diharapkan sistem CHRIS dapat mendukung operasional perusahaan secara lebih efisien, aman, dan terukur, serta memberikan kontribusi nyata dalam peningkatan kualitas pengelolaan sumber daya manusia di PT Ganda Visi Jayatama.

% MAKSUD DAN TUJUAN HARUS MENJAWAB JUDUL
% Program kerja magang ini dilaksanakan dengan maksud untuk menerapkan berbagai 
% \textit{hardskill} dan \textit{softskill} yang telah diperoleh selama masa 
% perkuliahan ke dalam lingkungan kerja profesional. Adapun tujuan dari program 
% kerja magang ini adalah untuk memperluas dan memperdalam \textit{hardskill} 
% melalui berbagai tugas yang diemban, serta mengembangkan \textit{softskill} 
% dalam berkoordinasi dan bekerja sama sebagai bagian dari tim. Secara spesifik, 
% tujuan pelaksanaan kerja magang ini adalah untuk berkontribusi dalam 
% pengembangan \textit{Human Resource Information System} (HRIS) pada 
% PT Ganda Visi Jayatama.





%-----------------------------------------------------------------------------%
\section{Waktu dan Prosedur Pelaksanaan Kerja Magang}
%-----------------------------------------------------------------------------%

Pelaksanaan kerja magang berlangsung dari tanggal 13 Januari 2025 sampai dengan 
13 Juli 2025 berdasarkan kontrak kerja yang telah disepakati dengan perusahaan. Selama periode magang ini dibimbing oleh seorang pembimbing lapangan yaitu Bapak Edo Setiawan yang menjabat sebagai Head Of Development di PT Ganda Visi Jayatama. Jadwal kerja magang di PT Ganda Visi Jayatama diatur sebagai berikut:

\begin{enumerate}
    \item Aktivitas kerja magang dilaksanakan setiap hari Senin hingga Jumat, 
    dengan jam kerja mulai pukul 09.00 WIB sampai dengan 18.00 WIB.
    \item Pelaksanaan kerja magang dilakukan secara \textit{Work From Office} (WFO).
\end{enumerate}

Selama menjalani program kerja magang, terdapat sejumlah prosedur yang telah ditetapkan, antara lain:

\begin{enumerate}
    \item Mengikuti sesi orientasi (\textit{onboarding}) pada minggu pertama kerja magang.
    \item Melakukan presensi harian dengan mencatat tugas yang telah diselesaikan 
    pada hari sebelumnya (\textit{yesterday tasks}), rencana aktivitas untuk hari ini 
    (\textit{today tasks}), serta kendala yang dihadapi dalam pengerjaan sebelumnya (\textit{blocking}).
    \item Berpartisipasi dalam rapat mingguan yang diadakan setiap hari Jumat 
    untuk membahas perkembangan proyek HRIS yang sedang dikerjakan.
    \item Menghadiri pertemuan bulanan untuk mendiskusikan pengembangan 
    \textit{boilerplate} perusahaan.
    \item Berkomunikasi dengan sesama karyawan melalui \textit{platform Discord}.
\end{enumerate}
